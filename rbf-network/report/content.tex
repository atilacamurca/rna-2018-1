% pandoc-fignos: caption name
\renewcommand{\figurename}{Figura}

\section{Introdução}

Redes RBF (do Inglês \emph{Radial Basis Function Networks}) são redes
que podem ser utilizadas para resolver problemas de Regressão e
Classificação. Esse tipo de rede pode aprender a aproximar uma tendência
usando várias curvas Gaussianas.

Uma rede RBF possui um ou mais camadas Escondidas conectadas a camada de
Entrada. A partir daí, obtém-se a saída da camada Escondida e realiza
uma soma ponderada através de OLAM (do Inglês \emph{Optimal Linear
Associative Memory}).

\section{Problemas}

\subsection{Íris}

O problema da Íris é a classificação de uma espécie de flor. Essa base
de dados é formada por 3 categorias: Setosa, Versicolor e Virgínica,
onde:

\begin{itemize}
\tightlist
\item
  Setosa é classificada como classe {[}1 0 0{]}, com 50 itens na base;
\item
  Versicolor é classificada como classe {[}0 1 0{]}, com 50 itens na
  base;
\item
  Virgínica é classificada como classe {[}0 0 1{]}, com 50 itens na
  base;
\end{itemize}

\subsection{Coluna Vertebral}

O problema da Coluna Vertebral é a classificação para identificar se um
paciente sofre ou não de algum problema na coluna. Essa base de dados
possui as seguintes categorias: Hernia, Spondylolisthesis e Normal,
onde:

\begin{itemize}
\tightlist
\item
  Hernia é classificada como classe {[}1 0 0{]}, com 60 itens na base;
\item
  Spondylolisthesis é classificada como classe {[}0 1 0{]}, com 150
  itens na base;
\item
  Normal é classificada como classe {[}0 0 1{]}, com 100 itens na base;
\end{itemize}

\section{Resultados}

Os resultados de ambos problemas foram bem abaixo do esperado,
provavelmente por um problema na elaboração do algoritmo. Seguem abaixo
os resultados apresentando a matriz de confusão para as realizações com
maior taxa de acerto.

\subsection{Íris}

Pior resultado:

\begin{verbatim}
Pesos =

     6.4735    78.6244   -98.9655    22.8229     9.3454
    -3.2905  -119.7229   133.0127   -17.1255    -1.8099
    -4.1831    41.0985   -34.0472    -5.6974    -7.5355

====  Sumário  =====
Num. Pred corretas: 0 de 30
        Realização: 2
Matriz de Confusão: [0 0 10;8 0 0;12 0 0]
    Taxa de acerto: 0
\end{verbatim}

Melhor resultado:

\begin{verbatim}
Pesos =

     5.8920   -45.6733   281.4740  -307.4812    82.1001
    -4.1660    22.8844  -191.2339   207.2235   -45.7574
    -2.7261    22.7889   -90.2401   100.2577   -36.3427

====  Sumário  =====
Num. Pred corretas: 19 de 30
        Realização: 20
Matriz de Confusão: [8 0 0;0 11 0;0 11 0]
    Taxa de acerto: 63.3333
\end{verbatim}

Sumário:

\begin{verbatim}
====  Sumário Geral  ====
        Acurácia: 26.8333
   Desvio Padrão: 18.5584
\end{verbatim}

\subsection{Coluna Vertebral}

Pior resultado:

\begin{verbatim}
Pesos =

 -3.1867    -4.9651    -6.3159    16.5321    -5.1169    12.6423  -16.6953
-31.6224 -2174.6339  4287.5155 -3182.3625 -1754.8589  2175.0692  601.4591
 33.8090  2179.5990 -4281.1997  3165.8303  1759.9758 -2187.7114 -584.7637

====  Sumário  =====
Num. Pred corretas: 15 de 62
        Realização: 5
Matriz de Confusão: [0 6 1;1 8 27;1 11 7]
    Taxa de acerto: 24.1935
\end{verbatim}

Melhor resultado:

\begin{verbatim}
Pesos =

  8.5008  23.3674   -53.6846   -73.4490   -19.4075    57.3331    80.2903
-51.0027  657.8846  729.6177   335.9711 -2032.2789  1504.9526 -1273.4145
 41.5018 -681.2524 -675.9331  -262.5218  2051.6862 -1562.2856  1193.1245

====  Sumário  =====
Num. Pred corretas: 37 de 62
        Realização: 12
Matriz de Confusão: [0 13 0;0 37 0;0 12 0]
    Taxa de acerto: 59.6774
\end{verbatim}

Sumário:

\begin{verbatim}
====  Sumário Geral  ====
        Acurácia: 41.0484
   Desvio Padrão: 11.0448
\end{verbatim}

\section{Conclusão}

Devido a possíveis problemas na implementação e a falta de um algoritmo
de seleção de parâmetros, por exemplo busca em grade com validação
cruzada de k-partes, os resultados obtidos foram muito abaixo do
esperado. Será necessário rever todo o algoritmo para identificar onde
deve ser modificado para um melhor resultado.

Repositório com código-fonte:
\url{https://github.com/atilacamurca/rna-2018-1}

Link para download:
\url{https://github.com/atilacamurca/rna-2018-1/archive/master.zip}
