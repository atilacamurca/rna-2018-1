% pandoc-fignos: caption name
\renewcommand{\figurename}{Figura}

\section{Introdução}

Redes RBF (do Inglês \emph{Radial Basis Function Networks}) são redes
que podem ser utilizadas para resolver problemas de Regressão e
Classificação. Esse tipo de rede pode aprender a aproximar uma tendência
usando várias curvas Gaussianas.

Uma rede RBF possui um ou mais camadas Escondidas conectadas a camada de
Entrada. A partir daí, obtém-se a saída da camada Escondida e realiza
uma soma ponderada através de OLAM (do Inglês \emph{Optimal Linear
Associative Memory}).

\section{Problemas}

\subsection{Íris}

O problema da Íris é a classificação de uma espécie de flor. Essa base
de dados é formada por 3 categorias: Setosa, Versicolor e Virgínica,
onde:

\begin{itemize}
\tightlist
\item
  Setosa é classificada como classe {[}1 0 0{]}, com 50 itens na base;
\item
  Versicolor é classificada como classe {[}0 1 0{]}, com 50 itens na
  base;
\item
  Virgínica é classificada como classe {[}0 0 1{]}, com 50 itens na
  base;
\end{itemize}

\subsection{Coluna Vertebral}

O problema da Coluna Vertebral é a classificação para identificar se um
paciente sofre ou não de algum problema na coluna. Essa base de dados
possui as seguintes categorias: Hernia, Spondylolisthesis e Normal,
onde:

\begin{itemize}
\tightlist
\item
  Hernia é classificada como classe {[}1 0 0{]}, com 60 itens na base;
\item
  Spondylolisthesis é classificada como classe {[}0 1 0{]}, com 150
  itens na base;
\item
  Normal é classificada como classe {[}0 0 1{]}, com 100 itens na base;
\end{itemize}

\section{Resultados}

Os resultados de ambos problemas foram dentro do esperado, usando 40\%
da base de teste como os centros e um sigma de \texttt{0.15}. Seguem os
resultados apresentando a matriz de confusão para as realizações com
maior taxa de acerto.

\subsection{Íris}

No problema da Íris os resultados encontrados encontram-se acima de
90,00\% na taxa de acerto.

Pior resultado:

\begin{verbatim}
====  Sumário  =====
Num. Pred corretas: 27 de 30
        Realização: 7
Matriz de Confusão: [10 0 0;0 7 1;0 2 10]
    Taxa de acerto: 90
\end{verbatim}

Melhor resultado:

\begin{verbatim}
====  Sumário  =====
Num. Pred corretas: 30 de 30
        Realização: 16
Matriz de Confusão: [10 0 0;0 9 0;0 0 11]
    Taxa de acerto: 100
\end{verbatim}

Sumário:

\begin{verbatim}
====  Sumário Geral  ====
        Acurácia: 95.1667
   Desvio Padrão: 2.7519
\end{verbatim}

\subsection{Coluna Vertebral}

No problema da oluna Vertebral os resultados encontrados encontram-se
acima de 77,00\% na taxa de acerto.

Pior resultado:

\begin{verbatim}
====  Sumário  =====
Num. Pred corretas: 48 de 62
        Realização: 5
Matriz de Confusão: [6 0 7;0 23 3;4 0 19]
    Taxa de acerto: 77.4194
\end{verbatim}

Melhor resultado:

\begin{verbatim}
====  Sumário  =====
Num. Pred corretas: 56 de 62
        Realização: 20
Matriz de Confusão: [10 0 4;0 36 1;1 0 10]
    Taxa de acerto: 90.3226
\end{verbatim}

Sumário:

\begin{verbatim}
====  Sumário Geral  ====
        Acurácia: 84.5161
   Desvio Padrão: 3.6028
\end{verbatim}

\section{Conclusão}

Devido a falta de um algoritmo de seleção de parâmetros, por exemplo
busca em grade com validação cruzada de k-partes, os resultados obtidos
foram dentro do esperado mas poderiam ser melhores e mais eficientes.

Repositório com código-fonte:
\url{https://github.com/atilacamurca/rna-2018-1}

Link para download:
\url{https://github.com/atilacamurca/rna-2018-1/archive/master.zip}
