\section{Introdução}\label{introduuxe7uxe3o}

Similar ao Perceptron Simples na característica de possuir apenas uma
camada, o ADALINE se diferencia no tipo de problema a ser resolvido, no
caso problemas de regressão. A atualização dos pesos sinápticos é feita
da mesma forma, entretanto não é usado uma função de ativação, a saída é
calculada diretamente de:

\[
u = w^T x = \sum^p_{i = 0} w_i x_i
\]

A ideia é que os pesos se ajustem de tal forma que crie uma reta o mais
próximo possível de todos os pontos da base de treinamento.

\section{Problemas}\label{problemas}

\subsection{Artificial I}\label{artificial-i}

Problema com uma variável independente e uma dependente, tal que:

\[f(x) = ax + b\]

A função escolhida foi \(f(x) = 2x + 3\), e ruído com valores entre
\(0.25\) e \(0.5\).

\subsection{Artificial II}\label{artificial-ii}

\section{Resultados}\label{resultados}

\subsection{Artificial I}\label{artificial-i-1}

Seja \(y = w_0 x_0 + w_1 x_1\) e \(x_0 = -1\), a partir dos dados da
realização abaixo podemos ver que os pesos estão bem próximos do
esperado:

\begin{align}
w_0 = 3.3746 &\approx 3 \\
w_1 = 1.9960 &\approx 2
\end{align}

\begin{verbatim}
pesos =

  -3.3746   1.9960

====  Sumário  =====
   Realização: 2
   MSE Treino: 0.00484138581311522
  RMSE Treino: 0.06958

desejado: 3.1623, calculado: 3.2738
desejado: 2.2016, calculado: 2.2657
desejado: 3.5498, calculado: 3.5964
desejado: 4.6203, calculado: 4.5239
desejado: 1.6037, calculado: 1.7012
desejado: 5.0342, calculado: 5.0884
desejado: 3.8168, calculado: 3.7577
desejado: 4.2099, calculado: 4.2013
desejado: 4.9089, calculado: 4.8061
desejado: 5.1364, calculado: 5.2497
desejado: 2.1618, calculado: 2.1044
desejado: 2.0717, calculado: 2.0238
desejado: 3.6192, calculado: 3.5158
desejado: 4.0313, calculado: 4.04
desejado: 4.4894, calculado: 4.6045
desejado: 4.7112, calculado: 4.6449
desejado: 4.5589, calculado: 4.4836
desejado: 4.6462, calculado: 4.7658
desejado: 2.9897, calculado: 2.9916
desejado: 3.1593, calculado: 3.1932

    MSE Teste: 0.00484138581311522
   RMSE Teste: 0.06958
\end{verbatim}

\subsection{Artificial II}\label{artificial-ii-1}

\section{Conclusão}\label{conclusuxe3o}

Dado que o ruído adicionado a função não seja muito alto, e que os
valores da base estejam normalizados, o ADALINE demonstra bons
resultados em problemas de regressão. Se os valores decimais fossem
ignorados a taxa de acerto no problema Artificial I, por exemplo, seria
de 100\% na realização 2, o que mostra sua alta precisão, dependendo
apenas da modelagem do problema.
