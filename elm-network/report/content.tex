% pandoc-fignos: caption name
\renewcommand{\figurename}{Figura}

\section{Introdução}

ELM, do Inglês \emph{Extreme Learning Machine}, é um novo algoritmo de
aprendizagem para redes neurais de camada escondida única. Em comparação
com o algoritmo de aprendizado de redes neurais convencionais, ele
supera a velocidade lenta de treinamento e os problemas de ajuste
excessivo. O ELM é baseado na teoria de minimização de riscos empíricos
e seu processo de aprendizado precisa de apenas uma única iteração. O
algoritmo evita várias iterações e minimização local. Ele tem sido usado
em vários campos e aplicativos por causa da melhor capacidade de
generalização, robustez e controlabilidade e rápida taxa de aprendizado.

\section{Problemas}

\subsection{Íris}

O problema da Íris é a classificação de uma espécie de flor. Essa base
de dados é formada por 3 categorias: Setosa, Versicolor e Virgínica,
onde:

\begin{itemize}
\tightlist
\item
  Setosa é classificada como classe {[}1 0 0{]}, com 50 itens na base;
\item
  Versicolor é classificada como classe {[}0 1 0{]}, com 50 itens na
  base;
\item
  Virgínica é classificada como classe {[}0 0 1{]}, com 50 itens na
  base;
\end{itemize}

\subsection{Coluna Vertebral}

O problema da Coluna Vertebral é a classificação para identificar se um
paciente sofre ou não de algum problema na coluna. Essa base de dados
possui as seguintes categorias: Hernia, Spondylolisthesis e Normal,
onde:

\begin{itemize}
\tightlist
\item
  Hernia é classificada como classe {[}1 0 0{]}, com 60 itens na base;
\item
  Spondylolisthesis é classificada como classe {[}0 1 0{]}, com 150
  itens na base;
\item
  Normal é classificada como classe {[}0 0 1{]}, com 100 itens na base;
\end{itemize}

\section{Resultados}

Os resultados de ambos problemas foram dentro do esperado, usando 50
neurônios na camada escondida. Seguem os resultados apresentando a
matriz de confusão para as realizações com maior taxa de acerto.

\subsection{Íris}

No problema da Íris os resultados encontrados encontram-se acima de
86,00\% na taxa de acerto.

Pior resultado:

\begin{verbatim}
====  Sumário  =====
Num. Pred corretas: 26 de 30
        Realização: 12
Matriz de Confusão: [10 0 0;0 11 1;0 3 5]
    Taxa de acerto: 86.6667
\end{verbatim}

Melhor resultado:

\begin{verbatim}
====  Sumário  =====
Num. Pred corretas: 30 de 30
        Realização: 11
Matriz de Confusão: [8 0 0;0 13 0;0 0 9]
    Taxa de acerto: 100
\end{verbatim}

Sumário:

\begin{verbatim}
====  Sumário Geral  ====
        Acurácia: 95.3333
   Desvio Padrão: 3.9589
\end{verbatim}

\subsection{Coluna Vertebral}

No problema da oluna Vertebral os resultados encontrados encontram-se
acima de 79,00\% na taxa de acerto.

Pior resultado:

\begin{verbatim}
====  Sumário  =====
Num. Pred corretas: 49 de 62
        Realização: 15
Matriz de Confusão: [6 0 6;0 27 4;3 0 16]
    Taxa de acerto: 79.0323
\end{verbatim}

Melhor resultado:

\begin{verbatim}
====  Sumário  =====
Num. Pred corretas: 58 de 62
        Realização: 17
Matriz de Confusão: [11 0 2;0 37 1;1 0 10]
    Taxa de acerto: 93.5484
\end{verbatim}

Sumário:

\begin{verbatim}
====  Sumário Geral  ====
        Acurácia: 86.8548
   Desvio Padrão: 3.598
\end{verbatim}

\section{Conclusão}

Devido a falta de um algoritmo de seleção de parâmetros, por exemplo
busca em grade com validação cruzada de k-partes, os resultados obtidos
foram dentro do esperado mas poderiam ser melhores e mais eficientes.

Repositório com código-fonte:
\url{https://github.com/atilacamurca/rna-2018-1}

Link para download:
\url{https://github.com/atilacamurca/rna-2018-1/archive/master.zip}
