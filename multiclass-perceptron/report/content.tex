% pandoc-fignos: caption name
\renewcommand{\figurename}{Figura}

\section{Introdução}

O perceptron simples é capaz de resolver problemas binários, que podem
ser separados linearmente. Mas para problemas com mais de uma classe é
necessário adaptar o algoritmo, adicionando mais neurônios, em que cada
um prediz uma classe.

\section{Problemas}

\subsection{Íris}

O problema da Íris é a classificação de uma espécie de flor. Essa base
de dados é formada por 3 categorias: Setosa, Versicolor e Virgínica,
onde:

\begin{itemize}
\tightlist
\item
  Setosa é classificada como classe {[}1 0 0{]}, com 50 itens na base;
\item
  Versicolor é classificada como classe {[}0 1 0{]}, com 50 itens na
  base;
\item
  Virgínica é classificada como classe {[}0 0 1{]}, com 50 itens na
  base;
\end{itemize}

\section{Resultados}

\subsection{Íris}

No problema da Íris os resultados encontrados encontram-se acima de
80,00\% na taxa de acerto. Nesse cenário a Matrix de Confusão foi a
seguinte:

\[
\begin{bmatrix}
10 & 0 & 0 \\
3 & 4 & 3 \\
0 & 0 & 10
\end{bmatrix}
\]

Nos melhores testes, foram a Matrix de Confusão:

\[
\begin{bmatrix}
10 & 0 & 0 \\
0 & 10 & 0 \\
0 & 0 & 10
\end{bmatrix}
\]

Taxa de acerto: 100.00\%.

De forma geral, a acurácia foi de 92,00\% com desvio padrão de 5,9628\%.

\section{Conclusão}

Apesar de elementar, o Perceptron Simples é um ótimo algoritmo de
classificação binária. Dado qualquer problema, ele é capaz de encontrar
uma regra de aprendizagem que garante encontrar uma solução ótima num
número finito de iterações.

Repositório com código-fonte:
\url{https://github.com/atilacamurca/rna-2018-1}

Link para download:
\url{https://github.com/atilacamurca/rna-2018-1/archive/master.zip}
